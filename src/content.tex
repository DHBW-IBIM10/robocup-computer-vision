\title{\papertitle}

\author{Rocco Schulz and Robert Wawrzyniak}
	
\publishers{Corporate State University\\Baden-Wuerttemberg - Stuttgart}

\date{
\vspace{0.6cm}
provided on 6 May 2013\\
\vspace{0.6cm}
School of: Business\\
\vspace{0.6cm}
Program: International Business Information Management\\
\vspace{0.6cm}
Course: WWI2010I\\
\vspace{0.6cm}
}





% DOCUMENT
\renewcommand{\baselinestretch}{1.5}\normalsize
\begin{document}
\pagestyle{scrheadings}

% roman numerals
\renewcommand{\thepage}{\Roman{page}}
% page numbers centered on top:
\chead{\pagemark}
\cfoot{}

%----------------------------------------------------------------------------
% Title Page
%----------------------------------------------------------------------------

% no page numbering
\thispagestyle{empty}
% include title page
\begin{titlepage}
%\vspace*{\fill}
\begin{center}
\vspace{3mm}

\textbf{\textsc{\Large 	\papertitle}} \\
\vspace{1.5cm}

\papertype \\ % Type of paper (defined in header.tex)
\vspace{0.4cm}

provided on \today \\ % Date of provision
\vspace{0.4cm}

School of: Business  \\
\vspace{0.4cm}

Program: International Business Information Management \\ % Program (defined in header.tex)
\vspace{0.4cm}

Course: WWI2010I \\ % Course (defined in header.tex)
\vspace{1.5cm}

by \\
\paperauthor % Author (defined in header.tex)
\vspace{1cm}
							
Baden-Wuerttemberg Cooperative State University Stuttgart	\\


% Note of confidentiality is only included when `confidential` is set
% to `true` in header.tex
%\textbf{Confidential}\\%
%The content of the paper must not be made available to third parties without approval of the training company.\\


\end{center}
\vspace*{\fill}
\end{titlepage}


%  \begin{abstract}
%  \vspace{1.6cm}
% % \textbf{\abstractname}: 
%  \end{abstract}


%----------------------------------------------------------------------------
% Table of Contents
%----------------------------------------------------------------------------
\renewcommand{\baselinestretch}{1.4}\normalsize
\tableofcontents
\renewcommand{\baselinestretch}{1.5}\normalsize
\newpage


%----------------------------------------------------------------------------
% Abbreviations
%----------------------------------------------------------------------------
% List needs to be indexed after each change.
% This is done by executing the following command:
% ~$ makeindex [filename].nlo -s nomencl.ist -o [filename].nls
\printnomenclature
\addcontentsline{toc}{section}{List of Abbreviations}
\nomenclature{CV}{Computer Vision}


\newpage

%----------------------------------------------------------------------------
% List Of Tables
%----------------------------------------------------------------------------
\listoftables
\addcontentsline{toc}{section}{\listtablename}
\newpage


%----------------------------------------------------------------------------
% List of Listings
%----------------------------------------------------------------------------
%\lstlistoflistings
\listoflistings
\addcontentsline{toc}{section}{List of Listings}
\newpage

% Arabic numerals for page numbering
\renewcommand{\thepage}{\arabic{page}}

% Set page number to 1: 
\setcounter{page}{1} 


%----------------------------------------------------------------------------
% Intro
%----------------------------------------------------------------------------
\section{Introduction}
\label{sec:introduction}




\subsection{Objectives}
\label{sec:objectives}



\subsection{Methodology and Structure}
\label{sec:methodology}



%----------------------------------------------------------------------------
% Theoretical Foundation
%----------------------------------------------------------------------------
\newpage
\section{RoboCup}
Robots play soccer.
Environment is dynamic.

\subsection{RoboCup Leagues}

\subsection{Related Research Areas}




\newpage
\section{Computer Vision at RoboCup}

CV is used for all leagues, explain how.
SSL Vision is used for the small size league and wil be covered in separate
section.

\subsection{Centralized}
in small size league. centralized control and cv.

\subsection{Distributed / Autonomous}
for other leagues. each robot has own cv.




\newpage
\section{SSL Vision}
SSL Vision stuff.

\subsection{Application Overview}



\begin{itemize}
  \item Written in C++, uses multithreading and comes with a GUI for
  configuration and controlling outputs
  \item Architecture: Multi-camera stack with one thread for each camera, for each camera plugins can be enabled
  \item Image processing framework with a modular structure, consisting of multiple plugins and common interfaces for data transfer
  \item Input: raw video streams from various cameras (IEEE 1394 / DCAM) or filestreams for development and testing
  \item Output: post processed image data in the GUI, robots identified by id and their positions
  \item See
  \url{http://www.informatik.uni-bremen.de/agebv/downloads/published/zickler_rs_09.pdf}
\end{itemize}

\subsection{The Image Processing Stack}

\begin{itemize}
  \item each robot on the field is uniquely identifiable by a colored marker
  \item SSL-Vision needs to determine robots’ real-world locations (as opposed
        to location in the image), directions and IDs to get these data
        different algorithms/techniques need to be applied in real time
  \item color segmentation
  \item camera calibration
  \item pattern detection
\end{itemize}

\paragraph{Color segmentation}
color segmentation is done with plugins for:
\begin{itemize}
  \item color thresholding
  \item runlength-encoding 
  \item region extraction
  \item region sorting
\end{itemize}

\paragraph{Color thresholding}
color thresholding is done via a lookup table which maps from the input image’s
3d color space (YUV) to a unique color label which represents any of the marker
colors, the ball color or other desired colors (source code) YUV: Color space,
e.g. used in PAL and NTSC videos, Y represents lumination and U/V color

\paragraph{Runlength encoding}

\begin{itemize}
  \item the line by line runlength encoding is applied on the thresholded image to speed up the region extraction that follows (source code)
  \item simple data compression technique
  \item Algorithm for runlength encoding of images:
for each row of pixels:
remember pixel value and go to next pixel;
if pixel value equals previous pixel, merge pixels
  \item Example from Wikipedia:
“(...) screen containing plain black text on a solid white background. There will be many long runs of white pixels in the blank space, and many short runs of black pixels within the text. Let us take a hypothetical single scan line, with B representing a black pixel and W representing white:
WWWWWWWWWWWWBWWWWWWWWWWWWBBBWWWWWWWWWWWWWWWWWWWWWWWWBWWWWWWWWWWWWWW
If we apply the run-length encoding data compression algorithm to the above hypothetical scan line, we get the following:
12W1B12W3B24W1B14W”
  \item works well due to the previous color thresholding
\end{itemize}

\paragraph{Region extraction}

pixels are only connected horizontally (lines)\\
vertical connections needed to get regions (polygons)\\
region extraction uses tree-based union find algorithm to traverse the
runlength-encoded image and merge neighboring runs of similar colors. (plugin
source code, region extraction source code)\\

Union find algorithm:\\
consists of find() and union()\\
good explanation at:
\url{http://valis.cs.uiuc.edu/~sariel/teach/2004/b/webpage/lec/22_uf.pdf} \\

bounding boxes and centroids of all merged regions are then computed and finally
sorted by color and size\\

centroids will later be used for location determination





%----------------------------------------------------------------------------
% Closing
%----------------------------------------------------------------------------

\clearpage
\section{Conclusion and Outlook}
\label{sec:conclusion}
SSL Vision is pretty mature.



%----------------------------------------------------------------------------
% APPENDIX
%----------------------------------------------------------------------------
% Appendix sections need to be within the subappendices environment.
% Use the command \appsection{title} instead of \section to introduce each
% appendix. This will add each appendix to the list of appendices.

% sets the appendix environment and resets the section counters
\newpage \begin{appendices} 
\appendixtocon %adds an 'Appendices' entry to the toc

\appendixpage %prints the title on the page

\subsection*{\listappendixname}
%--------------------------------
% style of the \listofappendices command is defined in header.tex
\listofappendices

% begin appendices on a new page
\newpage

%start environment for subappendices, so that new sections are formatted as
%subsections of appendix
\begin{subappendices}
\renewcommand{\setthesubsection}{\arabic{subsection}:}%

\appsection{Source Code}
\label{apx:code-template}


% close the appendices environment
\end{subappendices}
\end{appendices}
